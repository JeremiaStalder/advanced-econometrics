\documentclass{article}

\usepackage[a4paper,width=155mm,top=25mm,bottom=25mm]{geometry}
\usepackage{xcolor}
\usepackage{amsmath,amsthm,amsfonts,amssymb,amscd}
\usepackage{lastpage}
\usepackage{enumerate}
\usepackage{fancyhdr}
\usepackage{mathrsfs}
\usepackage{graphicx}
\usepackage{listings}
\usepackage{hyperref}
\usepackage{inputenc}
\usepackage{geometry}
\setlength{\parindent}{1.5em}
\setlength{\parskip}{1em} 

\newcommand\independent{\protect\mathpalette{\protect\independenT}{\perp}}
\def\independenT#1#2{\mathrel{\rlap{$#1#2$}\mkern2mu{#1#2}}}
\definecolor{blue}{rgb}{0.33, 0.45, 0.58}
\linespread{1.15}
\usepackage[bottom,flushmargin,hang,multiple]{footmisc}
\geometry{ footnotesep=0.7cm}

\title{The Effect of Eligibility on Wealth}
\author{J. Cordier, L. Gorkun-Voevoda, D. Kündig, E.-J. Senn and J. Stalder}
\date{\today}

\begin{document}

\maketitle



\section{Introduction}

\textcolor{blue}{State purpose of study}

\newpage

\section{Data and Descriptives}

\textcolor{blue}{Describe data set: variables, summary statistics and associations (correlation, correlation matrices, densities etc)\\
1-1.5 pages}


\newpage
%%%%%%%%%%%%%%%%%%%%%%%%%%%%%%%%%%%%%%%%%%%%%%%%%%%%%%%%%%%%%%%%%%%%%%
\section{Empirical Strategy and Results}


\subsection{Av. Treatment Effect}

The average treatment effect (henceforth ATE) is an average partial effect for a binary explanatory variable. It has been introduced by the counterfactual framework of Rubin (1974), called the Rubin Causal Model. In this framework, each individual has an outcome with and without treatment where $y_0$ ($y_1$) denotes the potential outcome without (with) treatment and $w$ is a dummy for the treatment status. Formally, the ATE is defined as  
\begin{align}
    \tau_{ate} \equiv E(y_1 - y_0) =E(y_1) - E(y_0)
    \label{ate}
\end{align}
where no assumptions about the distribution on $y$ are imposed. Criticism about the ATE highlight that it averages across population and is therefore not appropriate for policy analysis.\footnote{Wooldridge, 2010, pp. 903-905} Hence, also other treatment effects have been of interest such as the average treatment effect on the treated (ATET), the conditional treatment effect (CATE) and the conditional treatment effect on the treated (CATET).\newline

According to equation \ref{ate}, the ATE describes the expected difference of the potential outcome under treatment and without treatment. Put it differently, the causal effect of a binary treatment on average. However, identification is generally possible when all parameters of interest can be couched as random variables for with observations are available. As formally shown in equation \ref{ate_ext}, data on its own does not allow for a causal interpretation of the ATE. A potential outcome of treatment for an individual who did not receive treatment is never observed and vice versa. To answer an counterfactual question, additional assumptions are required.\footnote{Lechner, 2020, pp. 11-12, 23-25} 
\begin{align}
    \tau_{ate} &= E(y_1) - E(y_0) \nonumber \\
    &= [E(y_1 | w=1) - E(y_0| w=1)] \cdot P(w_1) + [E(y_1 | w=0) - E(y_0| w=0)] \cdot P(w_0) \nonumber \\
    &= [\underbrace{E(y | w=1)}_{\text{identified}} - \underbrace{E(y_0| w=1)}_{\text{counterfactual}}] \cdot \underbrace{P(w_1)}_{\text{identified}} + [\underbrace{E(y_1 | w=0)}_{\text{counterfactual}} - \underbrace{E(y| w=0)}_{\text{identified}}] \cdot \underbrace{P(w_0)}_{\text{identified}}
    \label{ate_ext}
\end{align}

In this paper, the ATE describes the causal effect of eligibility for the 401(k) pension saving plan on average where the household wealth denotes the dependent variable. 


%%%%%%%%%%%%%%%%%%%%%%%%%%%%%%%%%%%%%%%%%%%%%%%%%%%%%%%%%%%%%%%%%%%%%%
\subsection{Mean Comparison}

\textcolor{blue}{Describe and underlying assumptions.\\
Discuss omitted variables and endogeneity problem of mean comparison}\\

The supposedly simplest estimator of the ATE is a mean comparison. When 
\begin{align*}
    E(y|w=1) &= E(y_1|w=1)=E(y_1) \text{   and} \\
    E(y|w=0) &= E(y_0|w=0)=E(y_0)
\end{align*}
holds, then the ATE equals the ATET:
\begin{align}
    \tau_{ate} = \tau_{att} = E(y|w=1) - E(y|w=0)
    \label{ate_atet}
\end{align}
The underlying assumption is that treatment is statistically independent of the potential outcomes ($y_0, y_1$) as with randomised treatment. The right hand side of equation \ref{ate_atet} is estimated by taking the difference of the sample average of $y$ for the treated observations and the sample average of $y$ for the control observations.\footnote{Wooldridge, 2010, p. 907} \\
Therefore, the difference-in-mean estimator is unbiased, consistent and asymptotically normal under the following two assumptions.\footnote{see Lechner, 2020, Experiments p. 4-5 and Wooldridge, 2010, p. 905}
\begin{enumerate}
    \item Random selection into treatment and control group: $y_1, y_0 \independent W$ such that $ E(y_0|w=1) = E(y_0|w=0)$.
    \item Stable unit treatment value assumption (SUTVA): The treatment of observation $i$ does only influence its own potential outcome $i$.
\end{enumerate}

Note that the ATET can be consistently estimated by mean comparison under weaker assumptions.\footnote{Instead of full randomisation, the treatment $w$ is only independent of $y_0$ such that there is no restriction on relationship between $w$ and $y_1$.(Wooldridge, 2010, p. 907)

\begin{align*}
    E(y|w=1) - E(y|w=0) &= E(y_0|w=1) - E(y_0|w=0) + E(y_1-y_0|w=1) \\
    &= [E(y_0|w=1) - E(y_0|w=0)] + \tau_{att} \\
    &= 0 + \tau_{att}
\end{align*}}

However, the randomisation of treatment is often infeasible. In our case, people are eligible for 401(k) when certain requirements are fulfilled. This leads to self-selection into treatment and violates the first underlying assumption. Therefore, the difference-in-means estimator does not consider relevant variables which control for the self-selection into treatment.

\textcolor{red}{The ATE according to the difference-in-mean estimator corresponds to the following simple regression equation}

\begin{align}
    y = \beta_{0} + \beta_{1}W + u
\end{align}
where $W$ (the treatment dummy) denotes the sole dependent variable.
This leads to an endogeneity problem as the treatment variable correlates with the error term $u$. Several control variables which influence the treatment and the outcome variable are contained in the error term. In our case, omitted variables include income, age and education amongst others. \newline
The mean-difference estimator of the ATE is given by the following.

\begin{align}
    \widehat{\tau_{ate}} = \frac{1}{N} \sum_{i=1}^{N} \left( y_i w_i \right) -  \frac{1}{M} \sum_{i=1}^{M} \left(y_i  (1-w_i) \right)
\end{align}

%%%%%%%%%%%%%%%%%%%%%%%%%%%%%%%%%%%%%%%%%%%%%%%%%%%%%%%%%%%%%%%%%%%%%%
\subsection{Cond. Independence Assumption}
\textcolor{blue}{Which estimator uses CIA and comment on plausibility}\newline

As mentioned above, assumptions are needed for a causal interpretation of the counterfactual model. Those are the following:\footnote{Lechner, 2020, CIA pp. 10-14; Wooldridge, 2010, p. 908-911}

\begin{enumerate}
    \item The \textit{Conditional Independence Assumption (CIA)} defines the required independent variables such that all variables that jointly influence the potential outcomes $(y_0, y_1)$ and the selection into treatment are observed.
    \item As aforementioned, \textit{SUTVA} requires that there is no unrepresented level of treatment and that there is no interaction between the treatments. Therefore, the treatment of one observation does only affect the potential outcome of that observation in the given state.
    \item According to the \textit{Common support} or overlap assumption, the probability of treatment conditional on any independent variable is strictly larger than 0 and strictly smaller than 1. Therefore, any observation could potentially be observed with and without treatment.
    \item Under the \textit{Exogeneity of Confounders} assumption, the treatment does not affect the confounders in a way that is associated to the outcome variable.
\end{enumerate}

Hereinafter, the CIA is brought into focus.
Formally, the CIA is given by $y_1, y_0 \independent W|X=x; $ $ \forall x \in X $. Thus, the independence is violated when one confounder is unobserved. For this reason, the CIA is also dubbed unconfoundedness or selection-on-observables.\footnote{Lechner, 2020, CIA p. 10-11} Further, this assumption is not testable but is rather based on the richness of the data and on the empirical problem under consideration. 
Various estimators are based on the CIA such as the Inverse Probability Weighting or the OLS. \newline

In case, variables affecting the eligibility and the wealth  of households need to be observed. Given the limited data set, the plausibility of the CIA is debatable. We have identified three areas of possibly unobserved confounders.
HIER BIN ICH MIR NOCH UNSICHER, VA BEI DEN JOB RELATED VARIABLES
\begin{itemize}
    \item Classical variables such as the IQ, ability and the saving propensity might have an impact on eligibility and wealth.
    More intelligent (and able) people have high bargaining power on the job market. Therefore, they might be able to choose between different job offers where 401(k) eligibility could positively influence this decision. Further, one can assume that more intelligent people also earn higher wages or are more able to invest money appropriately which leads to a higher wealth. \\
    Besides, people with a higher unobserved saving propensity would probably participate in tax-advantaged retirement savings if eligible and would be more inclined to work at a place offering 401(k) plans. A higher saving propensity would most likely also result in higher wealth.
    \item Also some job related variables such as sector, political orientation of the employer, number of employees and the financial situation of the company could be of relevance. If one sector is more inclined to offer 401(k) plans and at the same time the employees have higher wealth, then this sector would be confounding. For example, the banking sector is presumably more profit-driven than other sectors such that the employer affords to offer 401(k) plans and the employees have above average wages and wealth. Similarly, right-wing employers, a low number of employees and a bad financial situation of the company could lead to a lower probability of 401(k) offers by the employer. If this further leads to higher or lower household wealth, these variables would be considered to be confounders.
    \item The aggregation on the household level might be problematic as relevant information is lost. In the given data set, various variables such as the wealth variables are aggregated on a household level. However, the distribution of the household wealth could be vital for the ATE analysis. For example, high wealth could stem from two very different households: one could consist of a rich single earner and another of a large middle class family where everyone contributes to the household wealth. 
\end{itemize}



%%%%%%%%%%%%%%%%%%%%%%%%%%%%%%%%%%%%%%%%%%%%%%%%%%%%%%%%%%%%%%%%%%%%%%
\subsection{Parametric Model and Nonlinear Squares Model}

\textcolor{blue}{Estimate and compare underlying assumptions}

\textcolor{blue}{Source: DA2, CIA, p. 20ff}\\
Inverse Probability Weighting: \\
Identification:
\begin{align}
    ate &= E_x[\mu(1,x) - \mu(0,x)] \\
    &= E \left( \frac{w \cdot y}{p(x)} - \frac{(1-w) \cdot y}{1-p(x)}  \right)
\end{align}
Parametric estimator for IPW based on parametric model: logit/probit model\\
\begin{align}
    P(Y=1|X=x) = G(x\beta) \equiv p(x)
\end{align}
where $G(.)$ maps index to response probability. \\
The estimated propensity score is used as a weight for the mean differencing. The $ate$ estimator is given by:

\begin{align}
    \widehat{ate} &= N^{-1} \sum_{i=1}^{N} \left( \frac{w_i \cdot y_i}{\hat{p}(x_i)} - \frac{(1-w_i) \cdot y_i}{1-\hat{p}(x_i)}  \right) \\
    &= N^{-1} \sum_{i=1}^{N}  \frac{[w_i - \hat{p}(x_i)]y_i}{\hat{p}(x_i)[1-\hat{p}(x_i)]}  
\end{align}

In a first step, I calculate the propensity score by a logit regression with the treatment as the dependent variable. After that, the ATE is identified by weighted mean differencing. Hence, the IPW corrects for confounding by considering the probability of being treated given the observed covariates. The IPW has the advantage that no tuning parameters have to be chosen and that it is close to being asymptotically efficient. However, since the IPW directly applies the p(x), it is highly sensitive to large values and misspecifications in the propensity score. (cf. Huber, Lechner, Wunsch, 2013) Therefore, I reduce the sample to observations with satisfy $0.05 < \hat{p}(x_i) < 0.95$.
Given that the following assumptions are fulfilled and under some regularity conditions, the IPW is generally consistent. (Wooldridge, 2010, p. 909-910, 921)

\begin{itemize}
    \item Ignorability in the mean: $E(y_0|X, w) = E(y_0|w) \text{  and  } E(y_1|X, w) = E(y_1|w)$. Note that X is a vector of observed covariates. If $X$ contains sufficient information to determine the treatment, then $(y_0, y_1) \text{  and  } w$ are independent conditional on $X$. The treatment and $(y_0, y_1)$ are allowed to be correlated but not after partialling out $X$.\\ and
    \item Overlap: There is a non-negative probability of observing units in control or treatment group
    \item what is lemma 12.2?
\end{itemize}

%%%%%%%%%%%%%%%%%%%%%%%%%%%%%%%%%%%%%%%%%%%%%%%%%%%%%%%%%%%%%%%%%%%%%%
\subsection{Non-Parametric and Semi-Parametric Estimator}

\textcolor{blue}{Estimate and compare underlying assumptions}

%%%%%%%%%%%%%%%%%%%%%%%%%%%%%%%%%%%%%%%%%%%%%%%%%%%%%%%%%%%%%%%%%%%%%%
\subsection{Lasso Estimator}

\textcolor{blue}{Estimate}

%%%%%%%%%%%%%%%%%%%%%%%%%%%%%%%%%%%%%%%%%%%%%%%%%%%%%%%%%%%%%%%%%%%%%%
\subsection{Comparison}

\textcolor{blue}{Compare ATE estimates under all strategies}


\newpage
%%%%%%%%%%%%%%%%%%%%%%%%%%%%%%%%%%%%%%%%%%%%%%%%%%%%%%%%%%%%%%%%%%%%%%
\section{Conclusion}

\textcolor{blue}{Summarise for a policy maker}


\end{document}

